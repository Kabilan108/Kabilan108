% ========================================================================================
%   Resume in Latex
%   Author : Jake Gutierrez
%   Based off of: https://github.com/sb2nov/resume
%   License : MIT
% ========================================================================================

\documentclass[letterpaper,11pt]{article}

\usepackage{latexsym}
\usepackage[empty]{fullpage}
\usepackage{titlesec}
\usepackage{marvosym}
\usepackage[usenames,dvipsnames]{color}
\usepackage{verbatim}
\usepackage{enumitem}
\usepackage[hidelinks]{hyperref}
\usepackage{fancyhdr}
\usepackage[english]{babel}
\usepackage{tabularx}
\usepackage{fontawesome5}
\usepackage{multicol}
\setlength{\multicolsep}{-3.0pt}
\setlength{\columnsep}{-1pt}
\input{glyphtounicode}


% ------------------------------------- FONT OPTIONS -------------------------------------
% sans-serif
% \usepackage[sfdefault]{FiraSans}
\usepackage[sfdefault]{roboto}
% \usepackage[sfdefault]{noto-sans}
% \usepackage[default]{sourcesanspro}

% serif
% \usepackage{CormorantGaramond}
% \usepackage{charter}


% -------------------------------------- PAGE SETUP --------------------------------------
\pagestyle{fancy}
\fancyhf{}  % clear all header and footer fields
\fancyfoot{}
\renewcommand{\headrulewidth}{0pt}
\renewcommand{\footrulewidth}{0pt}

% Adjust margins
\addtolength{\oddsidemargin}{-0.6in}
\addtolength{\evensidemargin}{-0.5in}
\addtolength{\textwidth}{1.19in}
\addtolength{\topmargin}{-.7in}
\addtolength{\textheight}{1.4in}

\urlstyle{same}

\raggedbottom
\raggedright
\setlength{\tabcolsep}{0in}

% Sections formatting
\titleformat{\section}{
  \vspace{-4pt}\scshape\raggedright\large\bfseries
}{}{0em}{}[\color{black}\titlerule \vspace{-5pt}]

% Ensure that generate pdf is machine readable/ATS parsable
\pdfgentounicode=1


% ------------------------------------ CUSTOM COMMANDS -----------------------------------
\newcommand{\resumeItem}[1]{
  \item\small{
    {#1 \vspace{-2pt}}
  }
}

\newcommand{\classesList}[4]{
    \item\small{
        {#1 #2 #3 #4 \vspace{-2pt}}
  }
}

\newcommand{\resumeSubheading}[4]{
  \vspace{-2pt}\item
    \begin{tabular*}{1.0\textwidth}[t]{l@{\extracolsep{\fill}}r}
      \textbf{#1} & \textbf{\small #2} \\
      \textit{\small#3} & \textit{\small #4} \\
    \end{tabular*}\vspace{-7pt}
}

\newcommand{\resumeSubheadingRow}[7]{
  \vspace{-2pt}\item
    \begin{tabular*}{1.0\textwidth}[t]{l@{\extracolsep{\fill}}r}
      \textbf{#1} & \textbf{\small #2} \\
      \textit{\small#3} & \textit{\small #4} \\
      \textit{\small#5} & \textit{\small #6} \\
      \textit{\small#7} \\
    \end{tabular*}\vspace{-7pt}
}

\newcommand{\resumeSubSubheading}[2]{
    \item
    \begin{tabular*}{0.97\textwidth}{l@{\extracolsep{\fill}}r}
      \textit{\small#1} & \textit{\small #2} \\
    \end{tabular*}\vspace{-7pt}
}

\newcommand{\resumeProjectHeading}[2]{
    \item
    \begin{tabular*}{1.001\textwidth}{l@{\extracolsep{\fill}}r}
      \small#1 & \textbf{\small #2}\\
    \end{tabular*}\vspace{-7pt}
}

\newcommand{\resumeSubItem}[1]{\resumeItem{#1}\vspace{-4pt}}

\renewcommand\labelitemi{$\vcenter{\hbox{\tiny$\bullet$}}$}
\renewcommand\labelitemii{$\vcenter{\hbox{\tiny$\bullet$}}$}

\newcommand{\resumeSubHeadingListStart}{\begin{itemize}[leftmargin=0.0in, label={}]}
\newcommand{\resumeSubHeadingListEnd}{\end{itemize}}

\newcommand{\reusmeSubHeadingListSmallStart}{\begin{itemize}[itemsep=-4pt, parsep=4pt]}
\newcommand{\resumeSubHeadingListSmallEnd}{\end{itemize}}

\newcommand{\resumeItemListStart}{\begin{itemize}}
\newcommand{\resumeItemListEnd}{\end{itemize}\vspace{-5pt}}


% ========================================================================================
%                                   RESUME STARTS HERE
% ----------------------------------------------------------------------------------------

\begin{document}

% --------------------------------------- HEADING ----------------------------------------

\begin{center}
    {\LARGE \scshape Tony Kabilan Okeke} \\
    \vspace{1pt}
    Current School Address -- Philadelphia, PA \\
    Home Address -- Enugu, Nigeria \\
    \vspace{1pt}
    \href{mailto:tony.x.okeke@gsk.com}{\faEnvelope\  tony.x.okeke@gsk.com}
    ~~
    \href{mailto:tko35@drexel.edu}{\faEnvelope\  tko35@drexel.edu}
    ~~
    \href{https://linkedin.com/in//t-k-o}{\faLinkedin\ linkedin.com/in/t-k-o}
    % ~~
    % \href{https://github.com/Kabilan108}{\faGithub\ Kabilan108}
    \vspace{-12pt}
\end{center}


% -------------------------------------- EDUCATION ---------------------------------------

\section{Education}
\resumeSubHeadingListStart
\resumeSubheadingRow
{Drexel University}
{Philadelphia, Pennsylvania}
{B.S. and M.S. in Biomedical Engineering (Accelerated Program)}
{Anticipated Graduation: June 2024}
{Concentrations: Bioinformatics \& Neuroengineering}
{Cummulative GPA: 4.00}
{Minor: Computer Science}
\resumeSubHeadingListEnd


% -------------------------------------- EXPERIENCE --------------------------------------

\section{Experience}
\resumeSubHeadingListStart

\resumeSubheading
{IVIVT - Non-Clinical Safety - Global Investigative Safety, GSK}
{April 2023 -- Present}
{Scientific Student Worker}
{Collegeville, Pennsylvania}

\vspace{-10pt}
\resumeItemListStart
\resumeItem{
    Utilizing \textit{SquidPy}, \textit{Seurat}, and other tools to enhance interpretation
    of GeoMx and 10X Visium datasets, and provide critical insights into spatial gene
    expression patterns and their implications in genetic toxicology research.
}
\resumeItem{
    Developing and implementing bioinformatics algorithms to analyze genomic datasets from
    non-clinical safety studies, providing insights into molecular mechanisms leading to
    potential safety concerns during drug development.
}
\resumeItem{
    Collaborated with eSTAR carcinogenicity working group to design a neural network model
    predicting molecular initiating events related to liver carcinogenicity using rat
    transcriptomics studies.
}
\resumeItemListEnd
\vspace{-2pt}

\resumeSubheading
{IVIVT - Non-Clinical Safety - Global Investigative Safety, GSK}
{April 2022 -- September 2022}
{Scientific Student Worker}
{Collegeville, Pennsylvania}

\vspace{-10pt}
\resumeItemListStart
\resumeItem{
    Utilized the \textit{Dash} and \textit{Flask} libraries in python to develop
    an interactive web application for performing statistical analysis on data
    generated in high-content imaging toxicology studies. Also utilized
    \textit{Dask} to parallelize computations on larger data sets to improve
    efficiency. The web application was then deployed via \textit{RStudio Connect}
    on the internal HPC.
}
\resumeItem{
    Trained the \textit{Noise2Void} deep neural network on HPC to perform noise
    reduction in microscopy images in order to improve the accuracy of CellProfiler
    pipelines for image segmentation and feature extraction.
}
\resumeItem{
    Developed a python package with tools for performing statistical analysis,
    visualization, and machine learning on high-content imaging data sets.
}
\resumeItem{
    Implemented pipelines in \textit{CellProfiler} and \textit{Columbus} to
    perform feature extraction for high-content images generated via cell painting
    assays.
}
\resumeItem{
    Utilized \textit{Scikit-learn} to implement decision tree, random forest, and
    support vector machine models for biomarker discovery on high-content imaging
    datasets.
}
\resumeItemListEnd
\vspace{-2pt}

\resumeSubheading
{Invenio Lab, Hospital of the University of Pennsylvania}
{March 2021 -- August 2022}
{Immunology Research Assistant}
{Philadelphia, Pennsylvania}

\vspace{-10pt}
\resumeItemListStart
\resumeItem{
    Developed SOPs and conducted assays for the isolation and extraction of DNA,
    RNA, and protein from human blood and urine samples, as well as the
    preparation of Next-Generation Sequencing libraries for Reduced Representation
    Bisulfite Sequencing (RRBS) and gene expression microarrays.
}
\resumeItem{
    Utilized \emph{Scikit-learn}, \emph{Pandas} and \emph{NumPy} to apply
    unsupervised learning algorithms to clinical and multi-omic datasets, and
    presented results to colleagues using \emph{Seaborn} in \emph{Jupyter}
    notebooks.
}
\resumeItem{
    Developed python and R scripts for analyzing DNA methylation levels in data
    from Illumina microarrays.
}
\resumeItem{
    Developed \emph{R} scripts for analyzing protein expression and clinical data
    from electronic medical records.
}
\resumeItem{
    Performed differential methylation, KEGG pathway enrichment, and Gene Ontology
    analysis on microarray results for patients who underwent cardiopulmonary
    bypass surgeries using \emph{bash} and \emph{R} scripts.
}
\resumeItemListEnd
\vspace{-2pt}

\resumeSubheading
{Zhou Lab, Children's Hospital of Philadelphia}
{May 2020 -- June 2021}
{Undergraduate Research Intern}
{Philadelphia, Pennsylvania}

\vspace{-10pt}
\resumeItemListStart
\resumeItem{
    Contributed to the development of R packages for analyzing DNA methylation
    levels in data from Illumina microarrays.
}
\resumeItem{
    Validated R package performance using \emph{GEO} public datasets.
}
\resumeItemListEnd

\resumeSubHeadingListEnd
\vspace{-18pt}


% --------------------------------------- PROJECTS ---------------------------------------

\section{Projects}
\vspace{-5pt}

\resumeSubHeadingListStart

\resumeProjectHeading
{\textbf{MEDDIBIA} $|$ \emph{Python, TensorFlow, Flask, Flutter}} {March 2023}
\resumeItemListStart
\resumeItem{
    Recipient of the Collaborative Team Award at the 2023 Phily CodeFest.
}
\resumeItem{
    Developed a machine learning-based solution for diagnosing skin conditions using
    pre-trained models like VGG16 and EfficientNet, achieving approximately 70\% accuracy
    in image-based classification.
}
\resumeItem{
    Employed GPT-3 for symptom identification from user input, and trained a
    \textit{Random Forest} classifier to predict the correspoding disease with 87\%
    accuracy; GPT-3 was then used to provide informative disease descriptions to users.
}
\resumeItem{
    Built a cross-platform app using Flask API for the backend, deployed on Heroku, and
    developed the frontend with Flutter, ensuring accessibility across multiple devices.
}
\resumeItemListEnd
\vspace{-10pt}

\resumeProjectHeading
{\textbf{MLGO, Deep Learning for GO Terms} $|$ \emph{Python, TensorFlow}} {September 2022}
\resumeItemListStart
\resumeItem{
    Develop a \textit{R} pipeline for preprocessing RNA-Seq datasets and performing
    differential expression analysis on data sets from \textit{GEO} and \textit{DEE2}.
}
\resumeItem{
    Developed neural network for predicting Gene Ontology enrichment in differential
    expression analysis data from RNA-Seq experiments.
}
\resumeItem{
    Trained an autoencoder model to reconstruct log fold-change values from RNA-Seq
    datasets, and used the latent space representation as input to a neural network
    for predicting GO terms.
}
\resumeItem{
    Utilized \textit{Tensorboard} and \textit{Keras Tuner} to monitor model performance
    and tune hyperparameters.
}
\resumeItemListEnd
\vspace{-10pt}

\resumeProjectHeading
{\textbf{CaBiD, Cancer Biomarker Discovery Tool} $|$ \emph{Python, Flask, Qt}} {September 2022}
\resumeItemListStart
\resumeItem{
    Developed a web application and GUI to investigate variations in gene expression across
    various cancer types.
}
\resumeItem{
    Preprocessed and curated datasets from GEO (Gene Expression Omnibus) and CuMiDa
    (Curated Microarray Database) in a SQLite database.
}
\resumeItem{
    Identified key differences in gene expression between healthy controls and tumoral
    samples across various cancer types.
}
\resumeItemListEnd
\vspace{-15pt}

\resumeProjectHeading
{\textbf{ELISA Analysis Tool} $|$ \emph{R, Shiny}} {September 2021}
\resumeItemListStart
\resumeItem{
    Processed Optical Density values from microplate readers using \emph{tidyverse}
    packages.
}
\resumeItem{
    Developed \emph{R} script for fitting OD values for ELISA standards to a
    5-Parameter logistic regression model to estimate unknown sample
    concentrations.
}
\resumeItem{
    Built interactive web-application for ELISA curve fitting using the
    \emph{RShiny} framework.
}
\resumeItemListEnd
\vspace{-16pt}

\resumeSubHeadingListEnd
\vspace{-3pt}

% ---------------------------------------- SKILLS ----------------------------------------

\section{Technical Skills}
\begin{itemize}[leftmargin=0.25in, label={}]
    \item\small {
        \textbf{Programming Languages: }{
            Python, R, Bash, C++, MATLAB, SQL, AWK, Git, PHP
        } \\
        \textbf{Frameworks and Libraries: }{
            TensorFlow, Keras, PyTorch, Scikit-learn, Flask, FastAPI, Shiny, Dash
        } \\
        \textbf{Bioinformatics Tools: }{
            Seurat, SquidPy, CellProfiler, Columbus
        } \\
        \textbf{Wet Lab Skills: }{
            PCR, qPCR, ELISA, Western Blot, DNA/RNA/Protein Extraction,
            NGS Library Preparation
        } \\
    }
\end{itemize}
\vspace{-18pt}

% ------------------------------------- PUBLICATIONS--------------------------------------

\section{Publications}

\reusmeSubHeadingListSmallStart
\item\small {
    Laudanski, Krzysztof, et al. "Unbiased analysis of temporal changes in immune serum
    markers in acute COVID-19 infection with emphasis on organ failure, anti-viral
    treatment, and demographic characteristics."
    \href{https://www.frontiersin.org/articles/10.3389/fimmu.2021.650465/full}
    {Frontiers in Immunology 12 (2021): 650465.}
}

\item\small {
    Laudanski, Krzysztof, et al. "Longitudinal urinary biomarkers of immunological
    activation in covid-19 patients without clinically apparent kidney disease versus
    acute and chronic failure."
    \href{https://www.nature.com/articles/s41598-021-99102-5.pdf?proof=tr}
    {Scientific Reports 11.1 (2021): 19675}.
}

\item\small {
    Laudanski, Krzysztof, et al. "Dynamic changes in central and peripheral neuro-injury
    vs. Neuroprotective serum markers in COVID-19 are modulated by different types of
    anti-viral treatments but do not affect the incidence of late and early strokes."
    \href{https://doi.org/10.3390/biomedicines9121791}
    {Biomedicines 9.12 (2021): 1791.}
}

\item\small {
    Laudanski, Krzysztof, et al. "A disturbed balance between blood complement protective
    factors (FH, ApoE) and common pathway effectors (C5a, TCC) in acute COVID-19 and
    during convalesce."
    \href{https://doi.org/10.1038/s41598-022-17011-7}
    {Scientific Reports 12.1 (2022): 13658.}
}

\item\small {
    Laudanski, Krzysztof, et al. "Persistent Depletion of Neuroprotective Factors
    Accompanies Neuroinflammatory, Neurodegenerative, and Vascular Remodeling Spectra in
    Serum Three Months after Non-Emergent Cardiac Surgery."
    \href{https://doi.org/10.3390/biomedicines10102364}
    {Biomedicines 10.10 (2022): 2364}.
}

\resumeSubHeadingListSmallEnd

\vspace{-20pt}

\section{Conference Abstracts}

\reusmeSubHeadingListSmallStart
\item\small {
    174: LONGITUDINAL CHANGES OF NEURO-SPECIFIC SERUM PROTEINS IN COVID-19 PATIENTS
    \href{https://doi.org/10.1097/01.ccm.0000807020.73561.ed}
    {Society of Critical Care Medicine 51st Critical Care Congress, April 2022}.
}

\item\small {
    181: PATTERNS OF URINARY BIOMARKERS OF IMMUNOLOGIC ACTIVATION AND SEPSIS IN PATIENTS WITH COVID-19
    \href{https://doi.org/10.1097/01.ccm.0000807048.74105.59}
    {Society of Critical Care Medicine 51st Critical Care Congress, April 2022}.
}

\resumeSubHeadingListSmallEnd

\vspace{-20pt}

% -------------------------------------- AWARD -------------------------------------------

\section{Awards and Honors}
\reusmeSubHeadingListSmallStart
\item\small {
    HESI GTTC Professional Development Award, Spring 2023
}

\item\small {
    Philly Codefest 2023 - Collaborative Team Award
}

\resumeSubHeadingListSmallEnd

\vspace{-20pt}


% ------------------------------------- SERVICE/LEADERSHIP -------------------------------

\section{Service and Leadership}

\reusmeSubHeadingListSmallStart
\item\small {
    Project Manager, \emph{Drexel Computational Design}, ~~May 2023 -- Present
}
\item\small {
    Vice President, \emph{Drexel Computational Design}, ~~March 2021 -- May 2023 \\
}
\item\small {
    Member, \emph{Tau Beta Pi}, ~~December 2021 -- Present
}
\item\small {
    Member, \emph{Drexel Society of Artificial Intelligence}, ~~September 2022 -- Present
}
\resumeSubHeadingListSmallEnd


\end{document}

% ----------------------------------------------------------------------------------------
%                                    RESUME ENDS HERE
% ========================================================================================
